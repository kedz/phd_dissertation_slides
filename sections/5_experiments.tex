\subsection{Experiments}


\begin{frame}{Baseline Linearization Strategies $\pi$}

    \textbf{Random (\textsc{Rnd})}
    \begin{itemize}
        \item Attribute-values are randomly ordered.
    \end{itemize}~\\
\end{frame}
\begin{frame}{Baseline Linearization Strategies $\pi$}
    \textbf{Increasing Frequency (\textsc{IncFreq})}
    \begin{itemize}
        \item Attribute-values are ordered in increasing frequency of 
            their occurrence in the training data.\[\operatorname{count}(x_i) \le \operatorname{count}(x_{i+1})\]
    \end{itemize}~\\

\end{frame}

\begin{frame}{Baseline Linearization Strategies $\pi$}
\begin{itemize}\item Random and Increasing Frequency linearization strategies 
            \alert{\textbf{DO NOT}} lead to controllable generation!\\~\\~\\
        
        \item Utterance realization order is determined implicitly by decoder language model.
    \end{itemize}
\end{frame}

%\begin{frame}{Evaluation Measures}
%    \textbf{Automatic Quality Measures}
%    \begin{itemize}
%        \item \textsc{Bleu} 
%        \item see paper for \textsc{Rouge} and others
%     \end{itemize}
%\end{frame}
%
%\begin{frame}{Evaluation Measures}
%    \textbf{Semantic Measures}
%    \begin{itemize}
%        \item \textsc{Semantic Error Rate (SER)} \[\textsc{SER} \triangleq \frac{\textrm{\#missing} + \textrm{\#wrong} + \textrm{\#added}   }{\textrm{\#attributes}}\]
%            Attribute-values automatically tagged + manual correction.\\~\\~\\
%    \end{itemize}
%\end{frame}
%\begin{frame}{Evaluation Measures}
%
%    \textbf{Control Measures}
%    \begin{itemize}
%        \item Order Accuracy (OA) \[\textsc{OA} \triangleq \frac{\sum_{\mathbf{x} \in \mathcal{D}}\mathds{1}\left\{\textrm{generated utterance }\boldsymbol{\hat{\mathbf{y}}} \textrm{ follows plan } \mathbf{x}  \right\}  }{\vert\mathcal{D}\vert }\]
%    \end{itemize}
%\end{frame}
%
%
%
%\begin{frame}
%\Huge 
%\textbf{And now some results...}
%
%\end{frame}

\usetikzlibrary{patterns}
\pgfplotstableread[row sep=\\,col sep=&]{
    interval        & Rnd   & If    & BgUP & NUP \\
    biGRU           & 2.64 & 12.64 & 0.26 & 0.26  \\
    Trans.     & 1.06  & 0.66  & 0.1 & 0.1 \\
    BART            & 0.1  & 0.18  & 0.20 & 0.20\\
    }\EtoEdata

\pgfplotstableread[row sep=\\,col sep=&]{
    interval        & Rnd   & If    & BgUP & NUP \\
    biGRU           & 12.56 & 19.20 & 3.40 & 1.58  \\
    Trans.     & 9.62  & 7.50  & 4.68 & 2.70 \\
    BART            & 1.50  & 1.86  & 0.52 & 0.54\\
    }\mydata


    \begin{frame}{Linearization Strategy Comparison}
        \begin{center}
\begin{tikzpicture}
    \begin{axis}[
            ybar,
            bar width=.15cm,
            width=0.5\textwidth,
            height=.35\textwidth,
         %   legend style={at={(1,1)},
         %       anchor=north east,legend columns=1},
            symbolic x coords={biGRU,Trans.,BART},
            ymajorgrids={true},
            xtick=data,
            %nodes near coords,
            nodes near coords align={vertical},
            enlarge x limits=0.25,
            ymin=0,ymax=15,
            ylabel={Semantic Error Rate \%},
            title={E2E Challenge},
        ]
        \addplot[fill=Red,color=Red!40,postaction={pattern=dots}] table[x=interval,y=If]{\EtoEdata};
        \addplot[fill=Blue!20,draw=Blue] table[x=interval,y=Rnd]{\EtoEdata};
%        \addplot[fill=Green!5,draw=Green,postaction={
%        pattern=north west lines,pattern color=Green
%    }] table[x=interval,y=BgUP]{\EtoEdata};
        \addplot[fill=Green!20,draw=Green,postaction={
        pattern=north east lines,pattern color=Green
    }] table[x=interval,y=NUP]{\EtoEdata};
     %   \legend{IncFreq, Rnd, AlignTrain+NUP}
    \only<2->{

        \node[xshift=6] (a1) at (axis cs:biGRU,5) {};
        \node[xshift=6] (a2) at (axis cs:biGRU,0.5) {};
        \draw[green,line width=0.6mm,->] (a1.center) -- (a2.center);

        \node[xshift=6] (b1) at (axis cs:Trans.,5) {};
        \node[xshift=6] (b2) at (axis cs:Trans.,0.5) {};
        \draw[green,line width=0.6mm,->] (b1.center) -- (b2.center);

        \node[xshift=6] (c1) at (axis cs:BART,5) {};
        \node[xshift=6] (c2) at (axis cs:BART,0.5) {};
        \draw[green,line width=0.6mm,->] (c1.center) -- (c2.center);
    }

    \end{axis}

    \begin{axis}[
            ybar,
            xshift=6.5cm,
            bar width=.15cm,
            width=0.5\textwidth,
            height=.35\textwidth,
            ymajorgrids={true},
            legend style={at={(1,1)}, nodes={scale=0.6, transform shape},
                anchor=north east,legend columns=-1},
            symbolic x coords={biGRU,Trans.,BART},
            xtick=data,
            %nodes near coords,
            nodes near coords align={vertical},
            ymin=0,ymax=20,
            enlarge x limits=0.25,
            %ylabel={Semantic Error Rate \%},
            title={ViGGO},
        ]
        \addplot[fill=Red,color=Red!40,postaction={pattern=dots}] table[x=interval,y=If]{\mydata};
        \addplot[fill=Blue!20,draw=Blue] table[x=interval,y=Rnd]{\mydata};
%        \addplot[fill=Green!5,draw=Green,postaction={
%        pattern=north west lines,pattern color=Green
%    }] table[x=interval,y=BgUP]{\mydata};
        \addplot[fill=Green!20,draw=Green,postaction={
        pattern=north east lines,pattern color=Green
    }] table[x=interval,y=NUP]{\mydata};
        \legend{IncFreq,Rnd,AlignTrain+NUP}
        \only<2->{
        \node[xshift=6] (a1) at (axis cs:biGRU,10) {};
        \node[xshift=6] (a2) at (axis cs:biGRU,2) {};
        \draw[green,line width=0.6mm,->] (a1.center) -- (a2.center);

        \node[xshift=6] (b1) at (axis cs:Trans.,10) {};
        \node[xshift=6] (b2) at (axis cs:Trans.,4) {};
        \draw[green,line width=0.6mm,->] (b1.center) -- (b2.center);

        \node[xshift=6] (c1) at (axis cs:BART,10) {};
        \node[xshift=6] (c2) at (axis cs:BART,2) {};
        \draw[green,line width=0.6mm,->] (c1.center) -- (c2.center);
    }
    \end{axis}
    \node[anchor=north west,align=left,text width=14cm] at (-2,-1) { 
        \small
        \begin{itemize} 
\item<2-> \textbf{Alignment Training reduces semantic error} compared to
other linearization strategies.
\item<3> Approaching \textbf{0 semantic errors} in large data (E2E Chal.) setting.
        \end{itemize}};

\end{tikzpicture}
\end{center}

\end{frame}

\pgfplotstableread[row sep=\\,col sep=&]{
    interval        & Rnd   & If    & BgUP & NUP \\
    biGRU           & 66.8 & 59.2 &66.4 & 66.3  \\
    Trans.     & 67.4  & 67.1 & 66.8 & 67.0 \\
    BART            & 66.5  & 65.6  & 66.2 & 66.6\\
    }\EtoEBleu

\pgfplotstableread[row sep=\\,col sep=&]{
    interval        & Rnd   & If    & BgUP & NUP \\
    biGRU           & 50.2 & 50.2 & 48.5 & 51.8  \\
    Trans.     & 52.0  & 52.3  & 48.7 & 51.6 \\
    BART            & 43.7  & 43.1  & 43.8 & 45.5\\
    }\ViggoBleu


    \begin{frame}{Linearization Strategy Comparison}
        \begin{center}
\begin{tikzpicture}
    \begin{axis}[
            ybar,
            bar width=.15cm,
            width=0.5\textwidth,
            height=.35\textwidth,
         %   legend style={at={(1,1)},
         %       anchor=north east,legend columns=1},
            symbolic x coords={biGRU,Trans.,BART},
            ymajorgrids={true},
            xtick=data,
            %nodes near coords,
            nodes near coords align={vertical},
            enlarge x limits=0.25,
            ymin=55,ymax=70,
            ylabel={BLEU},
            title={E2E Challenge},
        ]
        \addplot[fill=Red,color=Red!40,postaction={pattern=dots}] table[x=interval,y=If]{\EtoEBleu};
        \addplot[fill=Blue!20,draw=Blue] table[x=interval,y=Rnd]{\EtoEBleu};
%        \addplot[fill=Green!5,draw=Green,postaction={
%        pattern=north west lines,pattern color=Green
%    }] table[x=interval,y=BgUP]{\EtoEdata};
        \addplot[fill=Green!20,draw=Green,postaction={
        pattern=north east lines,pattern color=Green
    }] table[x=interval,y=NUP]{\EtoEBleu};
     %   \legend{IncFreq, Rnd, AlignTrain+NUP}
%    \only<2->{
%
%        \node[xshift=6] (a1) at (axis cs:biGRU,5) {};
%        \node[xshift=6] (a2) at (axis cs:biGRU,0.5) {};
%        \draw[green,line width=0.6mm,->] (a1.center) -- (a2.center);
%
%        \node[xshift=6] (b1) at (axis cs:Trans.,5) {};
%        \node[xshift=6] (b2) at (axis cs:Trans.,0.5) {};
%        \draw[green,line width=0.6mm,->] (b1.center) -- (b2.center);
%
%        \node[xshift=6] (c1) at (axis cs:BART,5) {};
%        \node[xshift=6] (c2) at (axis cs:BART,0.5) {};
%        \draw[green,line width=0.6mm,->] (c1.center) -- (c2.center);
%    }

    \end{axis}

    \begin{axis}[
            ybar,
            xshift=6.5cm,
            bar width=.15cm,
            width=0.5\textwidth,
            height=.35\textwidth,
            ymajorgrids={true},
            legend style={at={(0,1)}, nodes={scale=0.6, transform shape},
                anchor=north west,legend columns=-1},
            symbolic x coords={biGRU,Trans.,BART},
            xtick=data,
            %nodes near coords,
            nodes near coords align={vertical},
            ymin=40,ymax=55,
            enlarge x limits=0.25,
            %ylabel={Semantic Error Rate \%},
            title={ViGGO},
        ]
        \addplot[fill=Red,color=Red!40,postaction={pattern=dots}] table[x=interval,y=If]{\ViggoBleu};
        \addplot[fill=Blue!20,draw=Blue] table[x=interval,y=Rnd]{\ViggoBleu};
%        \addplot[fill=Green!5,draw=Green,postaction={
%        pattern=north west lines,pattern color=Green
%    }] table[x=interval,y=BgUP]{\mydata};
        \addplot[fill=Green!20,draw=Green,postaction={
        pattern=north east lines,pattern color=Green
    }] table[x=interval,y=NUP]{\ViggoBleu};
        \legend{IncFreq,Rnd,AlignTrain+NUP}
        \only<2->{
%        \node[xshift=6] (a1) at (axis cs:biGRU,10) {};
%        \node[xshift=6] (a2) at (axis cs:biGRU,2) {};
%        \draw[green,line width=0.6mm,->] (a1.center) -- (a2.center);
%
%        \node[xshift=6] (b1) at (axis cs:Trans.,10) {};
%        \node[xshift=6] (b2) at (axis cs:Trans.,4) {};
%        \draw[green,line width=0.6mm,->] (b1.center) -- (b2.center);
%
%        \node[xshift=6] (c1) at (axis cs:BART,10) {};
%        \node[xshift=6] (c2) at (axis cs:BART,2) {};
%        \draw[green,line width=0.6mm,->] (c1.center) -- (c2.center);
    }

    \end{axis}
    \node[anchor=north west,align=left,text width=14cm] at (-2,-1) { 
        \small
        \begin{itemize} 
            \item<2-> BART in small data setting (ViGGO) has lower \textsc{Bleu} 
    than trained from scratch models.
\item<3> But BART has lower Semantic Error! Be careful making judgements from \textsc{Bleu} alone!

        \end{itemize}};
    \uncover<2->{
            \node[draw=Violet,minimum height=2.3cm,minimum width=1.4cm,line width=0.5mm] at (11.0,0.3){};
        }
\end{tikzpicture}
\end{center}
\end{frame}




\pgfplotstableread[row sep=\\,col sep=&]{
    interval        & BgUP & NUP & Oracle \\
    biGRU           & 98.2 & 98.3 & 94.3  \\
    Trans.     & 99.9  & 100.0  & 95.0 \\
    BART            & 98.6 & 98.6 & 95.3 \\
    }\EtoEOA

\pgfplotstableread[row sep=\\,col sep=&]{
    interval        & BgUP   & NUP    & Oracle \\
    biGRU           & 89.8 & 93.7 &  92.2 \\
    Trans.     & 79.1 & 88.3 & 83.0  \\
    BART            & 98.3 & 98.2 & 97.2 \\
    }\ViggoOA


\begin{frame}{Control of Alignment Training + UtterancePlanner (UP)}
    \begin{center}
       % \resizebox{0.6\textwidth}{!}{
\begin{tikzpicture}
    \begin{axis}[
            ybar,
            bar width=.15cm,
            width=0.5\textwidth,
            height=.35\textwidth,
            symbolic x coords={biGRU,Trans.,BART},
            ymajorgrids={true},
            xtick=data,
            %nodes near coords,
            nodes near coords align={vertical},
            enlarge x limits=0.25,
            ymin=78,ymax=100,
            ylabel={Order Acc. (\%)},
            title={E2E},
        ]
        \addplot[fill=Red,color=Red!40,postaction={pattern=dots}] table[x=interval,y=BgUP]{\EtoEOA};
        \addplot[fill=Blue!20,draw=Blue] table[x=interval,y=NUP]{\EtoEOA};
%        \addplot[fill=Green!5,draw=Green,postaction={
%        pattern=north west lines,pattern color=Green
%    }] table[x=interval,y=BgUP]{\EtoEdata};
        \addplot[fill=Green!20,draw=Green,postaction={
        pattern=north east lines,pattern color=Green
    }] table[x=interval,y=Oracle]{\EtoEOA};
    \end{axis}

    \begin{axis}[
            ybar,
            xshift=6.5cm,
            bar width=.15cm,
            width=0.5\textwidth,
            height=.35\textwidth,
            legend style={at={(0.0,1)}, nodes={scale=0.60, transform shape},
                anchor=north west,legend columns=-1},
            symbolic x coords={biGRU,Trans.,BART},
            xtick=data,
            ymajorgrids={true},
            %nodes near coords,
            nodes near coords align={vertical},
            ymin=78,ymax=100,
            enlarge x limits=0.25,
            %ylabel={Order Acc. (\%) },
            title={ViGGO},
        ]
        \addplot[fill=Red,color=Red!40,postaction={pattern=dots}] table[x=interval,y=BgUP]{\ViggoOA};
        \addplot[fill=Blue!20,draw=Blue] table[x=interval,y=NUP]{\ViggoOA};
%        \addplot[fill=Green!5,draw=Green,postaction={
%        pattern=north west lines,pattern color=Green
%    }] table[x=interval,y=BgUP]{\mydata};
        \addplot[fill=Green!20,draw=Green,postaction={
        pattern=north east lines,pattern color=Green
    }] table[x=interval,y=Oracle]{\ViggoOA};
        \legend{BigramUP,NeuralUP,Oracle}
    \end{axis}

%\uncover<2>{

    \node[anchor=north west,align=left,text width=14cm] at (-2,-1) { 
        \small
        \begin{itemize} 
               \item<2-> On large data (E2E), models can follow BgUP or NUP plans over 98\% of the time!  
               \item<3-> Oracle (i.e. human) plans are harder to follow than model plans.
        \item<4->BART maintains controllability in both high and low data settings.

        \end{itemize}};
%\node[draw,fill=white,text width=8.5cm] (a)  at (4.5,-1.5) {\textbf{Alignment Training} Yields Consistent Reductions 
%in Semantic Error over other linearization strategies. };
%
%\draw[line width=0.5mm,->] (a) -- (4.80,0);
%\draw[line width=0.5mm,->] (a) -- (8.65,0);
%\draw[line width=0.5mm,->] (a) -- (1.0,0);
%\draw[line width=0.5mm,->] (a) -- (1.0,-4.8);
%\draw[line width=0.5mm,->] (a) -- (8.65,-4.9);
%\draw[line width=0.5mm,->] (a) -- (4.80,-4.6);
%}

\end{tikzpicture}
%}
\end{center}

%\begin{itemize}
%    \item<2->{ On large data (E2E), NLG models can follow BgUP or NUP plans over 98\% of the time!} 
%        \item<3->{ Oracle (i.e. human) plans are harder to follow than model plans.} 
%        \item<4->{BART maintains controllability in both high and low data settings.} 
%\end{itemize}

\end{frame}



%\begin{frame}{Alignment Training Yields Highly Controllable Model\\
%        In Large Data Setting}
%
%        \begin{tabular}{cccccc}
%            \toprule
%            & & \multicolumn{4}{c}{E2E} \\
%            \cmidrule(lr){3-6} 
%          Model & Plan & \textsc{Bleu} & \textsc{Rouge} & SER (\%)$\downarrow$ & OA (\%)$\uparrow$ \\
%            \midrule
%    biGRU & \textsc{BgUP} & 66.4 & 68.3 & \textbf{0.26} & 98.2 \\
%          & \textsc{NUP} & 66.3 & 68.9 & \textbf{0.26} & \textbf{98.3} \\
%          & \textsc{Oracle} & \textbf{69.3} & \textbf{77.3} & 0.84 & 94.3 \\
%            \midrule
%            {\footnotesize Transformer} & \textsc{BgUP} & 66.8 & 68.4 & \textbf{0.00} & 99.9 \\
%          & \textsc{NUP} & 67.0 & 69.0 & \textbf{0.00} & \textbf{100.0} \\
%          & \textsc{Oracle} & \textbf{69.3} & \textbf{77.0} & 0.76 & 95.0 \\
%%            \midrule
%%            BART & \textsc{BgUP} & 66.2 & 68.7 & \textbf{0.20} & \textbf{98.6} \\
%%             & \textsc{NUP} & 66.6 & 69.2 & \textbf{0.20} & \textbf{98.6} \\
%%          & \textsc{Oracle} & \textbf{68.3} & \textbf{77.1} & 0.70 & 95.3 \\
%            \bottomrule
%        \end{tabular}
%        \begin{itemize}
%     %       \item Little difference between \textsc{BgUP} and NUP on SER and OA.
%            \item Model plans are easier to follow than \textsc{Oracle} (i.e. human).
%        \end{itemize}
%\end{frame}
%
%\begin{frame}{Alignment Training Yields (Slightly Less) Control\\
%        In Small Data Setting}
%
%        \begin{tabular}{cccccc}
%            \toprule
%            & & \multicolumn{4}{c}{ViGGO} \\
%            \cmidrule(lr){3-6} 
%          Model & Plan & \textsc{Bleu} & \textsc{Rouge} & SER (\%)$\downarrow$ & OA (\%)$\uparrow$ \\
%            \midrule
%    biGRU & \textsc{BgUP} & 48.5 & 58.5 & 3.40 & 89.8 \\
%          & \textsc{NUP} & 51.8 & 62.6 & \textbf{1.58} & \textbf{93.7} \\
%          & \textsc{Oracle} & \textbf{54.1} & \textbf{65.5} & 2.42 & 92.2 \\
%            \midrule
%            BART & \textsc{BgUP} & 43.8 & 54.0 & \textbf{0.52} & \textbf{98.3} \\
%          & \textsc{NUP} & 45.5 & 57.6 & 0.54 & 98.2 \\
%          & \textsc{Oracle} & \textbf{47.1} & \textbf{60.4} & 0.82 & 97.2 \\
%%            \midrule
%%            BART & \textsc{BgUP} & 66.2 & 68.7 & \textbf{0.20} & \textbf{98.6} \\
%%             & \textsc{NUP} & 66.6 & 69.2 & \textbf{0.20} & \textbf{98.6} \\
%%          & \textsc{Oracle} & \textbf{68.3} & \textbf{77.1} & 0.70 & 95.3 \\
%            \bottomrule
%        \end{tabular}
%        \begin{itemize}
%           % \item Higher semantic error rate and lower order accuracy than on E2E Chal.
%            \item Trained from scratch models (biGRU) have higher \textsc{Bleu} and \textsc{Rouge} but more semantic errors than BART!
%           % \item Model plans are easier to follow than \textsc{Oracle} (i.e. human).
%        \end{itemize}
%\end{frame}


\begin{frame}{Random Permutation Stress Test}

    {\Large \textbf{Following Utterance Planner Model is relatively easy.}}\\
    \begin{itemize}
        \item Plans are essentially coming from the training distribution.
        \item Models have lower entropy than humans.
    \end{itemize}~\\

    {\Large \textbf{Can a model follow a random plan?}}
        \begin{itemize}
               \item Randomly generate plans i.e. random permutations of attribute-value sequences.
               \item Evaluate on Order Accuracy. 
           \end{itemize}
        

\end{frame}

%\begin{frame}{Training Distribution VS Permutation Distribution}
%    \begin{center}
%    \begin{tabular}{llcc}
%        \toprule
%        & & \multicolumn{2}{c}{SER (\%)$\downarrow$} \\
%            \cmidrule(lr){3-4} 
%            \multicolumn{2}{c}{Model} & E2E  & ViGGO \\
%            \midrule
%            biGRU& NUP & \textbf{0.22} & \textbf{9.62} \\
%                 & Rand  & 1.14          & 13.58 \\
%            \midrule
%            Transformer & NUP & \textbf{0.08} & \textbf{24.18} \\
%              &Rand &     0.78   &   28.34 \\
%            \midrule
%            BART & NUP & 0.64 & \textbf{1.34} \\
%                 &Rand &          \textbf{0.42} & 2.30 \\
%    \bottomrule
%\end{tabular}\end{center}
%    \begin{itemize}
%\item Random permutations \textbf{harder to follow} than NUP plan (except E2E BART!)\end{itemize}
%\end{frame}
%
%\begin{frame}{\textbf{Data Augmentation} in \textbf{Large Data} Setting (E2E)}
%    \resizebox{\textwidth}{!}{%
%    \begin{tikzpicture}
%        \node at (0,0) {
%    \begin{tabular}{llcc}
%        \toprule
%        & & \multicolumn{2}{c}{E2E} \\
%            \cmidrule(lr){3-4} 
%            \multicolumn{2}{c}{Model} & SER (\%)$\downarrow$   & OA (\%)$\uparrow$ \\
%            \midrule
%            \multicolumn{2}{l}{biGRU} & 1.14 & 94.44 \\
%                                      &\alert{+DataAug} & \textbf{0.54} & \textbf{97.34} \\
%            \midrule
%            \multicolumn{2}{l}{Transformer} & 0.78 & 95.20 \\
%                                            &\alert{+DataAug} & \textbf{0.40} & \textbf{98.10} \\
%            \midrule
%            \multicolumn{2}{l}{BART} & 0.42 & 97.78 \\
%                                     &\alert{+DataAug} & \textbf{0.22} & \textbf{98.78} \\
%    \bottomrule
%    \end{tabular}};
%
%    \node[text width=11cm,align=left,font=\Large,anchor=north] at (0,-3.15) { 
%\begin{itemize} \item \textbf{+DataAug improves representation} ~~~~~~~~ of arbitrary permutations. \end{itemize}};
%\end{tikzpicture}
%}
%\end{frame}
%
%\begin{frame}{\textbf{Data Augmentation} in \textbf{Small Data} Setting (ViGGO)}
%    \resizebox{\textwidth}{!}{%
%    \begin{tikzpicture}
%        \node at (0,0) {
%    \begin{tabular}{llcc}
%        \toprule
%        & & \multicolumn{2}{c}{ViGGO} \\
%            \cmidrule(lr){3-4} 
%            \multicolumn{2}{c}{Model} & SER (\%)$\downarrow$   & OA (\%)$\uparrow$ \\
%            \midrule
%            \multicolumn{2}{l}{biGRU} & \textbf{13.58} & 46.72 \\
%                            &+DataAug & 14.46 & \textbf{49.26} \\
%            \midrule
%            \multicolumn{2}{l}{Transformer} & 28.34 & \textbf{18.70} \\
%                                  &+DataAug & \textbf{25.72} & 18.10 \\
%            \midrule
%            \multicolumn{2}{l}{BART} & 2.30 & 82.00 \\
%                                             &\alert{+DataAug} & \textbf{1.82} & \textbf{87.98} \\
%    \bottomrule
%    \end{tabular}};
%
%    \node[text width=11cm,align=left,font=\Large,anchor=north] at (0,-3.15) { 
%        \begin{itemize}
%            %\item  Mixed results on trained from scratch models.
%        \item \textbf{BART+DataAug improves representation} 
%            of arbitrary permutations.
%    \end{itemize}
%        };
%\end{tikzpicture}
%}
%\end{frame}

\pgfplotstableread[row sep=\\,col sep=&]{
    interval        & NUP & Rand & DA \\
    biGRU           & 98.72 & 94.44 & 97.34  \\
    Trans.     & 99.64  & 95.20 & 98.10 \\
    BART            & 96.52 & 97.78 & 98.78 \\
    }\EtoEPermOA

\pgfplotstableread[row sep=\\,col sep=&]{
    interval        & NUP & Rand & DA\\
    biGRU           & 62.04 & 46.72 &  49.26 \\
    Trans.     & 31.34 & 18.70 & 18.10  \\
    BART            & 91.40 & 82.00 & 87.98 \\
    }\ViggoPermOA


\begin{frame}{Permutation Stress Test}
\begin{tikzpicture}
    \begin{axis}[
            ybar,
            bar width=.15cm,
            width=0.5\textwidth,
            height=.35\textwidth,
            symbolic x coords={biGRU,Trans.,BART},
            ymajorgrids={true},
            xtick=data,
            %nodes near coords,
            nodes near coords align={vertical},
            enlarge x limits=0.25,
            ymin=90,ymax=100,
            ylabel={Order Acc. (\%)},
            title={E2E},
        ]
        \addplot[fill=Red,color=Red!40,postaction={pattern=dots}] table[x=interval,y=NUP]{\EtoEPermOA};
        \addplot[fill=Blue!20,draw=Blue] table[x=interval,y=Rand]{\EtoEPermOA};
%        \addplot[fill=Green!5,draw=Green,postaction={
%        pattern=north west lines,pattern color=Green
%    }] table[x=interval,y=BgUP]{\EtoEdata};
      \only<2->{  \addplot[fill=Green!20,draw=Green,postaction={
        pattern=north east lines,pattern color=Green
    }] table[x=interval,y=DA]{\EtoEPermOA};}

\only<1>{
\node[xshift=3.0] (a1) at (axis cs:biGRU,98.72) {};
\node[xshift=3.0] (a2) at (axis cs:biGRU,94.44) {};
\node[xshift=3.0] (b1) at (axis cs:Trans.,99.64) {};
\node[xshift=3.0] (b2) at (axis cs:Trans.,95.20) {};
\draw[->,draw=Red,line width=0.5mm] (a1.center) -- (a2.center); 
\draw[->,draw=Red,line width=0.5mm] (b1.center) -- (b2.center); 
}

\only<2->{
\node[xshift=0.0] (a1) at (axis cs:biGRU,98.72) {};
\node[xshift=0.0] (a2) at (axis cs:biGRU,94.44) {};
\node[xshift=0.0] (b1) at (axis cs:Trans.,99.64) {};
\node[xshift=0.0] (b2) at (axis cs:Trans.,95.20) {};
\draw[->,draw=Red,line width=0.5mm] (a1.center) -- (a2.center); 
\draw[->,draw=Red,line width=0.5mm] (b1.center) -- (b2.center); 
}


    \end{axis}

    \begin{axis}[
            ybar,
            xshift=6.5cm,
            bar width=.15cm,
            width=0.5\textwidth,
            height=.35\textwidth,
            legend style={at={(0.0,1)}, nodes={scale=0.6, transform shape},
                anchor=north west,legend columns=1},
            symbolic x coords={biGRU,Trans.,BART},
            xtick=data,
            ymajorgrids={true},
            %nodes near coords,
            nodes near coords align={vertical},
            ymin=15,ymax=100,
            enlarge x limits=0.25,
            %ylabel={Order Acc. (\%) },
            title={ViGGO},
        ]
        \addplot[fill=Red,color=Red!40,postaction={pattern=dots}] table[x=interval,y=NUP]{\ViggoPermOA};
        \addplot[fill=Blue!20,draw=Blue] table[x=interval,y=Rand]{\ViggoPermOA};
%        \addplot[fill=Green!5,draw=Green,postaction={
%        pattern=north west lines,pattern color=Green
%    }] table[x=interval,y=BgUP]{\mydata};
       \only<2->{ \addplot[fill=Green!20,draw=Green,postaction={
        pattern=north east lines,pattern color=Green
    }] table[x=interval,y=DA]{\ViggoPermOA};}
        \legend{NeuralUP,Random,Random+DataAug}

\only<1>{
\node[xshift=3.0] (a1) at (axis cs:biGRU,62.04) {};
\node[xshift=3.0] (a2) at (axis cs:biGRU,46.72) {};
\node[xshift=3.0] (b1) at (axis cs:Trans.,31.64) {};
\node[xshift=3.0] (b2) at (axis cs:Trans.,18.70) {};

\node[xshift=3.0] (c1) at (axis cs:BART,91.40) {};
\node[xshift=3.0] (c2) at (axis cs:BART,82.00) {};
\draw[->,draw=Red,line width=0.5mm] (a1.center) -- (a2.center); 
\draw[->,draw=Red,line width=0.5mm] (b1.center) -- (b2.center); 
\draw[->,draw=Red,line width=0.5mm] (c1.center) -- (c2.center); 
}
\only<2>{
\node[xshift=0.0] (a1) at (axis cs:biGRU,62.04) {};
\node[xshift=0.0] (a2) at (axis cs:biGRU,46.72) {};
\node[xshift=0.0] (b1) at (axis cs:Trans.,31.64) {};
\node[xshift=0.0] (b2) at (axis cs:Trans.,18.70) {};

\node[xshift=0.0] (c1) at (axis cs:BART,91.40) {};
\node[xshift=0.0] (c2) at (axis cs:BART,82.00) {};
\draw[->,draw=Red,line width=0.5mm] (a1.center) -- (a2.center); 
\draw[->,draw=Red,line width=0.5mm] (b1.center) -- (b2.center); 
\draw[->,draw=Red,line width=0.5mm] (c1.center) -- (c2.center); 
}


    \end{axis}


\uncover<2>{
%\node[draw,fill=white,text width=8.5cm] (a)  at (4.5,-1.5) {\textbf{Alignment Training} Yields Consistent Reductions 
%in Semantic Error over other linearization strategies. };
%
%\draw[line width=0.5mm,->] (a) -- (4.80,0);
%\draw[line width=0.5mm,->] (a) -- (8.65,0);
%\draw[line width=0.5mm,->] (a) -- (1.0,0);
%\draw[line width=0.5mm,->] (a) -- (1.0,-4.8);
%\draw[line width=0.5mm,->] (a) -- (8.65,-4.9);
%\draw[line width=0.5mm,->] (a) -- (4.80,-4.6);
}
\end{tikzpicture}

\begin{itemize}
    \item Models struggle to follow random plans compared to Utterance Planner  ordering.
\item<2-> A phrase-based data augmentation improves the ability to follow difficult plans.
\end{itemize}


\end{frame}
%?
%?\begin{frame}{Random Permutation Stress Test}
%?    \begin{tabular}{llcccc}
%?        \toprule
%?        & & \multicolumn{2}{c}{E2E} & \multicolumn{2}{c}{ViGGO} \\
%?            \cmidrule(lr){3-4} \cmidrule(lr){5-6}
%?            \multicolumn{2}{c}{Model} & SER (\%)$\downarrow$   & OA (\%)$\uparrow$ & SER (\%)$\downarrow$   & OA (\%)$\uparrow$\\
%?            \midrule
%?            \multicolumn{2}{l}{biGRU} & 1.14 & 94.44 & \textbf{13.58} & 46.72 \\
%?                                      &+DataAug & \textbf{0.54} & \textbf{97.34} & 14.46 & \textbf{49.26} \\
%?            \midrule
%?            \multicolumn{2}{l}{Transformer} & 0.78 & 95.20 & 28.34 & \textbf{18.70} \\
%?                                            &+DataAug & \textbf{0.40} & \textbf{98.10} & \textbf{25.72} & 18.10 \\
%?            \midrule
%?            \multicolumn{2}{l}{BART} & 0.42 & 97.78 & 2.30 & 82.00 \\
%?                                     &+DataAug & \textbf{0.22} & \textbf{98.78} & \textbf{1.82} & \textbf{87.98} \\
%?    \bottomrule
%?    \end{tabular}
%?\end{frame}

\begin{frame}{Controllable Generation Takeaways}


\begin{itemize}
\item Can you make an arbitrary Seq2Seq model controllable? \\~\\
    \begin{itemize}
            
\item<2->{\alert{Yes! Alignment Training works regardless of architecture.}} \end{itemize}~\\ ~\\
            

%\item Are there differences between recurrent models or
%    transformers? \\ \uncover<3->{\textbf{Recurrent models fair better in smaller data settings.}}
%\item What about large pretrained models? \\ \uncover<4->{\textbf{These work great and you should use them when dataset is small.}}
%    \end{itemize}
%\end{itemize}
%\end{frame}

%\begin{frame}{Takeaways}
%\begin{itemize}
\item How robust is a controllable Seq2Seq model? \\~\\
    \begin{itemize}
        \item<3->{\alert{Performance drops on random  plans!}}\\~\\
\item<4->{\alert{Can be mitigated by data augmentation!}}\\~\\
\item<5->{\alert{BART maintains controllability in small/large data settings as well as when followng random permutation plans.}}\\
    \end{itemize}
\end{itemize}

\end{frame}

\begin{frame}{Phrase-based Data Augmentation}
    \resizebox{\textwidth}{!}{
        \begin{tikzpicture}

            \def\th{5mm};
            \def\td{2mm};
    \node[text height=\th,text depth=\td] (aromi) at (-5,-4) {Aromi};
            \node[text height=\th,text depth=\td] (is) at (-1.5,-4) {is};
            \node[text height=\th,text depth=\td] (not) at (0.0,-4) {not};

            \node[text height=\th,text depth=\td] (a) at (1.5,-4) {a};
            \node[text height=\th,text depth=\td] (ff) at (4,-4) {family-friendly};
            \node[text height=\th,text depth=\td] (est) at (6.5,-4) {establishment};


            \uncover<2->{
            \node (root) at (-2.5,0) {S};
            \node (rootNP) at (-5,-1) {NP};
            \node (rootNPNNP) at (-5,-2) {NNP};
            \only<1-3>{
            \node (r1c2) at (0.0,-1) {VP};
        }
            \only<4->{
                \node[color=mLightBrown] (r1c2) at (0.0,-1) {VP};

            }
            \only<1-2>{
                \node (r2c3) at (4,-2) {NP};
                \node (det) at (1.5,-3) {DET};
                \node (jj) at (4,-3) {JJ};
                \node (nn) at (6.5,-3) {NN};
            }
            \only<3->{
                \node[color=mLightBrown] (r2c3) at (4,-2) {NP};
                \node[color=mLightBrown] (det) at (1.5,-3) {DET};
                \node[color=mLightBrown] (jj) at (4,-3) {JJ};
                \node[color=mLightBrown] (nn) at (6.5,-3) {NN};
            }
            \only<1-3>{
                \node (r2c1) at (-1.5,-2) {VB};
                \node (r2c2) at (0.0,-2) {RB};
                \draw[-] (r1c2) -- (r2c1);
                \draw[-] (r1c2) -- (r2c2);
                \draw[-] (r1c2) -- (r2c3);
            }
            \only<4->{
                \node[color=mLightBrown] (r2c1) at (-1.5,-2) {VB};
                \node[color=mLightBrown] (r2c2) at (0.0,-2) {RB};
                \draw[mLightBrown,-] (r1c2) -- (r2c1);
                \draw[mLightBrown,-] (r1c2) -- (r2c2);
                \draw[mLightBrown,-] (r1c2) -- (r2c3);
            }


            \only<1-3>{
                \draw[-] (r2c1) -- (is);
            }
            \only<4->{
                \draw[color=mLightBrown,-] (r2c1) -- (is);

            }
            \draw[-] (rootNPNNP) -- (aromi);
            \draw[-] (root.south west) -- (rootNP.north east);
            \draw[-] (rootNP) -- (rootNPNNP);
            \draw[-] (root.south east) -- (r1c2.north west);
            \only<1-2>{
                \draw[-] (r2c3) -- (det);
                \draw[-] (r2c3) -- (jj);
                \draw[-] (r2c3) -- (nn);
                \draw[-] (det) -- (a);
                \draw[-] (jj) -- (ff);
                \draw[-] (nn) -- (est);
            }
            \only<3->{
                \draw[mLightBrown,-] (r2c3) -- (det);
                \draw[mLightBrown,-,color=mLightBrown] (r2c3) -- (jj);
                \draw[mLightBrown,-,color=mLightBrown] (r2c3) -- (nn);
                \draw[mLightBrown,-] (det) -- (a);
                \draw[mLightBrown,-] (jj) -- (ff);
                \draw[mLightBrown,-] (nn) -- (est);
            }
            \only<1-3>{
                \draw[-] (r2c2) -- (not);
            }
            \only<4>{
                \draw[color=mLightBrown,-] (r2c2) -- (not);
            }
        }
        

            \node[anchor=north west,align=left,inner sep=0,outer sep=0,text height=0mm,text width=11cm] at (0.7,0.5) {
                    \begin{enumerate}
                        \item<2-> Parse training examples.
                \item<3-> Create additional training examples from constituent phrases.
                \end{enumerate}};

        \end{tikzpicture}
    }

    \uncover<3->{
    \resizebox{\textwidth}{!}{
        {  \begin{minipage}{1.45\textwidth}
$\left[\!\!\!\left[\begin{array}{l} \textsc{Inform} \\ \textrm{family\_friendly=yes} \end{array} \right]\!\!\!\right]$ \texttt{["a", "family-friendly", "establishment"]}
\end{minipage}}}
}
\uncover<4>{
    \resizebox{\textwidth}{!}{
        {  \begin{minipage}{1.45\textwidth}
$\left[\!\!\!\left[\begin{array}{l} \textsc{Inform} \\  \textrm{family\_friendly=no} \end{array} \right]\!\!\!\right]$ \texttt{["is", "not", "a", "family-friendly", "establishment"]}
\end{minipage}}}}


\end{frame}

